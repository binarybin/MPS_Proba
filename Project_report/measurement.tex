
The \texttt{Measurement} class includes functions for calculating the expectation of a given operator, the conditional probability of certain states, and possibly the correlation functions of different time steps. Both the conditional probability and correlation function may require us to store the MPS at each step. By calling this storage, we don't need to rerun the MPS procedure to calculate new physical quantities. And this is one of the advantages of MPS over Monte Carlo method. In the simplest form, the user can specify a set of states/operators through a defined structure, which are then combined with final state of the system to give (conditional) probabilities. If time permitted, we will try to extend the measurement to include time-dependent information.
